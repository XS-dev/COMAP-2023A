%Volerra食饵_捕食者模型
$$
\begin{array}{r}
	\dot{x}(t)=x(r-a y)=r x-a x y \\
	\dot{y}(i)=y(-d+b x)=-d y+b x y
\end{array}
$$

%x:食饵数量 km^2
%y:捕食者数量 k
%r:食饵增长系数 
%a:捕食者掠取食饵能力系数
%d:捕食者独自存在时的死亡系数
%b:食饵对捕食者的供养能力系数                                                           c


%Volerra食饵_捕食者模型+logistic项
$$
\begin{aligned}
	& \dot{x}_1(t)=r_1 x_1\left(1-\frac{x_1}{N_1}-\sigma_1 \frac{x_2}{N_2}\right) \\
	& \dot{x}_2(t)=r_2 x_2\left(-1+\sigma_2 \frac{x_1}{N_1}-\frac{x_2}{N_2}-3m\right)
\end{aligned}
$$

%x1:草地面积 km^2
%x2:羊净产量 k
%r1:草地生长参数
%r2:羊生长参数
%N1:草地饱和产量 km^2
%N2:羊饱和产量(人为控制)k
%σ1:草的产量对羊的供养能力系数 
%σ2:羊的捕食能力系数
%m:龙的捕食量 k

%改进CASA模型估算NPP(净生产力)
$$
\begin{aligned}
	\operatorname{NPP}(x, t)= & \operatorname{FAPAR}(x, t) \times \varepsilon_{\max }(x, t) \times \\
	& T_{\varepsilon 1}(x, t) \times T_{\varepsilon 2}(x, t) \times W_{\varepsilon}(x, t)
\end{aligned}
$$

%植被吸收的光合有效辐射(FAPAR)W
%最大光能利用率(εmax)%
%温度胁迫系数(Tε1 、Tε2)
%水分胁迫系数(Wε)
$$
\begin{aligned}
r1\propto\operatorname{NPP}
\end{aligned}
$$
%草地生长参数正比于NPP